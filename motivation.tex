\noindent
\textbf{Motivation.}
As a supertask, infinite blockchains are not realizable in practice. Their study
is motivated as follows. First, while blockchain protocols have been modelled as
executions which are stopped and completed after finite polynomial time, this is
not actually what happens in the real protocol. The question of what happens if
a blockchain protocol is left to run \emph{ad infinitum} therefore comes up
naturally. Secondly, taking the case of infinite-time execution allows us to
separate the combinatorial properties of chain structures and study them
separately from the cryptographic mechanisms through which they are realized in
polynomial settings. This approach makes it necessary to remove instances of
cryptographic signatures or the Random Oracle model, as they depend on a
security parameter in which the execution must be polynomial. Therefore, the
approach highlights the bare essential properties of chains. Last, because there
is no security parameter involved, the treatment becomes simpler and more
direct. As such, Chernoff bound applications to polynomially growing structures
(in which results are proven \emph{with overwhelming probability}) are turned
into simpler Law of Large Number applications to infinitely sized structures (in
which results are proven \emph{almost surely}). While this somewhat deviates
from the concrete setting of realistic protocols, it gives rise to a novel
proving strategy for the study of blockchain protocols which we feel is
particularly instructive: Before studying a proposed protocol in the polynomial
setting, which involves many nuanced details, the proposed protocol can be
studied in the infinite setting. Although this process may not confirm that the
proposed protocol is secure, it can highlight attacks which would be
significantly more difficult to find in a polynomial-time treatment. Finally,
in our infinite setting, it suffices to show that good properties hold with
respect to the \emph{expected value} of our random variables, as concentration
is ensured by the law of large numbers. Our model therefore establishes a
theoretical justification for the widespread cryptographer practice of proof
sketches in which good properties are shown \emph{in expectation} instead of
\emph{with high probability}, but also highlights the shortcomings of this
approach.
