\begin{figure*}
\fbox{
	\begin{minipage}{0.9 \textwidth}
  The \emph{block tree} functionality is parameterized by the number of parties
  $n$, the number of adversarial parties $t$, the number of queries per
  round $q$, and the probability of query success $p$.

  The functionality maintains the following state variables:
  \begin{enumerate}
    \item $\blocks$:
          A counter indicating how many blocks have been mined until now.
    \item $\indices$:
          A set of triplets
          $(\party, i, b)$ where $\party \in [n]$ is the index of a party,
          $i \in [\blocks]$ is the
          index of a block in the view of the functionality, and $b$ is the index of a
          block in the view of party $\party$. The triplet indicates that party
          $\party$ uses $b$ to refer to block $i$.
    \item $\mail$:
          A set of pairs $(\party, \vec{m})$, where
          $\party \in [n]$ is the index of a party and $\vec{m}$ is a list of
          previously unseen blocks for $\party$.
    \item $\prevmail$:
          A set containing the mail sent during the previous round to be
          delivered at the current round.
    \item $\parent$:
          A set of tuples $(i', i)$ to maintain the
          parent-child relationship of the block tree, indicating that block
          $i'$ is the parent of block $i$.
    \item $\data$:
          A set of tuples $(i, \vec{x})$ to indicate
          that block $i$ contains data $\vec{x}$, where $\vec{x}$ is a sequence
          of transactions that are defined by the application.
    \item $\queries$:
          A set of tuples $(\party, q')$ containing the remaining queries
          $q' \in [q]$ of party $\party$ for the current round.
  \end{enumerate}

  At the beginning of the execution, set $\blocks$ to $1$ and $\indices$ to
  $\{\party \in [n]: (\party, 1, 1)\}$ (indicating that every party $\party$ has
  seen the genesis block and will use $1$ to refer to block $1$). Set
  $\mail(\party) = \emptyset$ for every party $\party$ and $\prevmail =
  \emptyset$. Set $\parent(0) = \bot$ to indicate that the genesis block has
  no parent and $\data(0) = \epsilon$ to indicate that the genesis block has no
  transactions.

	\end{minipage}
}
\caption{
The state of the \emph{block tree} functionality $\mathcal{F}_{BT}$
\label{fig:blocktree1}
}
\end{figure*}

\begin{figure*}
\fbox{
	\begin{minipage}{0.9 \textwidth}
  At the beginning of each new round, clear the mailboxes of
  all parties by setting $\prevmail = \mail$ and $\mail(\party) = \emptyset$ for
  every party $\party$ and reset the queries by setting
  $\queries = \{(\party, q): \party \in [n]\}$.

  Let $\fresh(\party) = \max\{b': \exists i': (\party', i', b') \in \indices\} +
  1$ indicate the next available block index in the view of party $\party$. Let
  $\lclgbl(\party, b)$ be $\bot$ if there is no $i$ such that $(\party, i, b) \in
  \indices$. Otherwise, let $\lclgbl(\party, b) = i$. Let
  $\gbllcl(\party, i)$ be $\bot$ if there is no $b$ such that
  $(\party, i, b) \in \indices$. Otherwise, let $\gbllcl(\party, i) = b$.

  Let $\party$ indicate the index of a party invoking a given method. The
  adversary can choose to invoke any method by assuming the identity of any
  corrupted party of her choosing.

  The functionality supplies the following methods:

	\begin{itemize}
      \item $\getmail()$:
            Returns $\prevmail(\party)$.
			\item $\mine(b, \vec{x})$:
            Attempt to mine a block with parent $b$ and application data
            $\vec{x}$.

            If $\queries(\party) = 0$, return $\bot$. Otherwise,
            decrement $\queries(\party)$.
            Let $i = \lclgbl(\party, b)$. If $i \neq \bot$, flip a
            biased random coin with distribution $\{0: 1 - p, 1: p\}$ and set
            its outcome to random variable $X$.

            If $X = 0$, then the return $\bot$, indicating that
            mining has been unsuccessful. Otherwise, increment
            $\blocks$ indicating that mining has been successful and
            place $(\party, \blocks, b')$ into
            $\indices$, where $b' = \fresh(\party)$.
            Set $\parent(\blocks) = i$ and $\data(\blocks) = \vec{x}$.
            Finally, return $b'$.
      \item $\getparent(b)$:
            Let $i = \lclgbl(\party, b)$. If $i = \bot$, return $\bot$.
            Otherwise, return $\gbllcl(\party, \parent(i))$.
      \item $\getdata(b)$:
            Let $i = \lclgbl(\party, b)$. If $i = \bot$, return $\bot$.
            Otherwise, return $\gbllcl(\party, \data(i))$.
      \item $\broadcast(b)$:
            For every party $\party' \neq \party$, invoke
            $\reveal(b, \party')$.
      \item $\reveal(b, \party')$:
            Let $i = \lclgbl(\party, b)$. If
            $i = \bot$ or $\gbllcl(\party', i) \neq \bot$,
            return $\bot$ (to ensure that party $\party'$ has
            not yet seen block $i$). Let
            $b' = \fresh(\party')$.
            Place the new triplet $(\party', i, b')$ into $\indices$ and
            set $\mail(\party') = \mail(\party') b'$.
            If $\parent(i) \neq \bot$ and $\gbllcl(\party', \parent(i)) = \bot$,
            invoke $\textsf{reveal}(\textsf{getparent}(i), \party')$.
      \item $\snoopmail(\party')$:
            If $\party$ is not adversarial, return $\bot$. Otherwise, return
            $\mail(\party')$.
      \item $\injectmail(\party', B)$:
            If $\party$ is not adversarial, return $\bot$.
            Check if a permutation of $\mail(\party')$ is a subsequence of $B$.
            If not, return $\bot$. Otherwise, set $\mail(\party') = B$.
  \end{itemize}

	\end{minipage}
}
\caption{
The methods of the \emph{block tree} functionality $\mathcal{F}_{BT}$
\label{fig:blocktree2}
}
\end{figure*}
