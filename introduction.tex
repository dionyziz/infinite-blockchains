\section{Introduction}

Blockchain protocols attempt to solve the problem of maintaining a decentralized
ledger. A population of nodes wishes to report an append-only ledger which
is consistent and irreversible among all honest parties, a property called
\emph{persistence}, and eventually contains all application data (\emph{transactions}) which the honest parties
attempt to inject into it, a property called \emph{liveness}. To achieve this,
each honest party locally maintains a \emph{chain} and attempts to \emph{mine}
on top of it to extend it~\cite{EC:GarKiaLeo15}. The probability of successfully
mining a block is assumed to be bounded. When a block is mined successfully, it
is broadcast to the rest of the nodes and the nodes adopt the longest blockchain
that they see.

This protocol gives rise to three properties that are attained by honest chains:
\begin{enumerate}
  \item \textbf{Chain growth.} The chain of each honest party grows at least by
        a certain rate.
  \item \textbf{Common prefix.} If a particular short suffix is removed from
        each honestly adopted chain, all the chains coincide.
  \item \textbf{Chain quality.} Any large enough portion of an honestly adopted
        chain contains at least some percentage of honestly generated blocks.
\end{enumerate}

The Common Prefix property ensures that, if a small amount of blocks is pruned
from the end of an honestly adopted chain, the remaining chain is \emph{stable}
and contains data which can be reported as part of a ledger. This property
ensures that the ledger built on top of the chain has \emph{persistence}:
any transaction reported as stable will eventually be included by all
parties in the same ledger position and will not
be reverted in the future. On the one hand, \emph{chain growth} and \emph{chain
quality} together ensure that the ledger has \emph{liveness} in that it eventually
reports honestly broadcast transactions in it.

These results have been established in the literature for
\emph{polynomially-bound} executions of blockchain protocols. However, real
blockchain protocol executions never terminate. Hence, the natural question
arises: Does the protocol still attain these properties if we leave it to run
\emph{ad infinitum}? Answering this question poses certain technical
difficulties which do not make it straightforward. First, we need to define
what it means for a party executing for an unbounded amount of time to \emph{adopt} an
infinitely-sized blockchain and to \emph{report} an infinitely-sized ledger,
given that the parties are modelled as Turing Machines which, at any finite
point in time, can only have used a finite amount of memory and therefore
adopted a finite blockchain and a finite ledger. Secondly, the basic tools used
to prove blockchains secure and to model proof-of-work have inherent
polynomial-time restrictions. The primary issue is that of
bounding proof-of-work queries. In classical proof-of-work treatments,
the limits on the computational majority are enforced using the Random Oracle
model. However, in our case, this model is inherently problematic, as
non-polynomial adversaries can break it and even honest infinitely-running
executions will inevitably run into collisions. More fundamentally, any use of a
security parameter is not possible in this context, as bad events which
happen with negligible probability in some parameter blow up when the execution
becomes exponential. This requires that we abstract the idea of block tree
maintenance into a separate ideal functionality and highlights the properties
required for work-based consensus.

\import{./}{motivation.tex}

\import{./}{related.tex}

\import{./}{contributions.tex}
