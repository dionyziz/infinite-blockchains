\begin{abstract}
  Blockchain protocols have been studied as the solution to the anonymous
  byzantine agreement problem in which proof-of-work is used to distinguish
  honest majority and establish liveness and persistence of a decentralized
  ledger with overwhelming probability among polynomially-bound parties. In such
  settings, it is posited that the execution of the protocol completes after
  polynomial time. This allows the use of standard cryptographic modelling
  techniques such as the Random Oracle.

  In this paper, we consider the setting in which non-polynomial nodes
  collaborate to establish \emph{liveness} (eventual inclusion) and
  \emph{persistence} (agreement) of a decentralized ledger under honest
  computational majority. Contrary to polynomially-bound executions, we consider
  chains which are growing unboundedly, and hence construct ledgers that have
  infinite elements. We bypass the need for the Random Oracle model, which can
  work only in polynomial-time settings, using a novel functionality which
  maintains a blocktree and captures the essential properties of a growing
  blockchain and bounded ``hashing power.'' We show that the standard
  \emph{backbone} blockchain protocol for the polynomially-bound
  setting is also a protocol for the unbounded setting. We put
  forth appropriate definitional extensions for chain properties and show that the
  (potentially infinite) chains produced by this protocol have the properties of
  \emph{chain growth}, \emph{common prefix} and \emph{chain quality} and give
  rise to the \emph{liveness} and \emph{persistence} properties of a
  (potentially infinite) decentralized ledger. Contrary to the
  polynomially-bound setting where results are proven with \emph{overwhelming
  probability}, our results are proven \emph{almost surely}.
\end{abstract}
