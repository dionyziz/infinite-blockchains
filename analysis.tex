\section{Analysis}

Given an unbounded execution, let $X^r$ for $r \in \mathbb{N}$ denote the random
variable indicating whether any honest party has been successful in round $r$.
Let $Y^r$ denote the random variable indicating whether only \emph{one} honest
party has been successful in round $r$. We call such rounds \emph{uniquely
successful}. Let $X^r_\party$ denote that party $\party$ has been successful
during round $r$. Note that, since the execution is unbounded, the number of
random variables we are considering here is countably infinite.

\begin{definition}[Typical execution]
  An execution of the protocol of Section~\ref{sec.construction} is
  \emph{typical} if:

  \begin{enumerate}
    \item For every honest party $\party$ and every round $r \in \mathbb{N}$,
          there exists some round $r' > r$ such that $X^{r'}_\party = 1$.
  \end{enumerate}
\end{definition}

\begin{definition}[Honest majority]
  An execution has \emph{honest majority} if the expectation of a uniquely
  successful round is more than the expectation of an adversarial round, i.e.,
  $\E[Y^r] > \E[Z^r]$.
\end{definition}

\begin{theorem}[Typicality]
  An execution of the protocol of Section~\ref{sec.construction} is typical
  \emph{almost surely}.
\end{theorem}
\begin{proof}
  Let $\party$ be an arbitrary honest party and $r \in \mathbb{N}$ be an
  arbitrary round. Consider the quantity
  $f = \lim_{r' \to \infty} \sum_{i = r + 1}^{r'} \frac{X^{i}_\party}{r' - r - 1}$.
  For any $i$, it holds that
  $\E[X^i_\party] \geq p$. From the Law of Large numbers, $\Pr[f =
  \E[X^i_\party]] = 1$, therefore $\Pr[f \geq 0] = 1$. From the probabilistic
  method, it follows that there exists some round $r' > 1$ such that
  $X^{r'}_\party = 1$.
  \qed
\end{proof}

\begin{theorem}[Chain growth]
  Consider a typical execution of the protocol of
  Section~\ref{sec.construction}. The protocol has chain growth.
\end{theorem}
\begin{proof}
  Suppose for contradiction that there exists an honest party $\party$ and a round $r$
  such that for all $r' > r$, it holds that
  $|\chain_\party^{r'}| \leq |\chain_\party^r|$. First, note that
  Line~\ref{alg.round:maxvalid} of Algorithm~\ref{alg.round} only allows the
  local chain to grow, so we will have that
  $|\chain_\party^{r'}| = |\chain_\party^r|$. If $\party$ is successful at any
  round $r' > r$, its chain will grow. Therefore we obtain that
  $\forall r' > r: X^{r'}_\party = 0$. This contradicts typicality.
  \qed
\end{proof}

Note that the above holds even when no honest majority exists.

\begin{theorem}[Common prefix]
  The protocol of Section~\ref{sec.construction} has Common prefix.
\end{theorem}
\begin{proof}
  \dionyziz{Argue that $\lim \frac{Y}{Z} > 1$ in typicality, then use this
  here...}
  \qed
\end{proof}

\begin{theorem}[Chain quality]
  The protocol of Section~\ref{sec.construction} has Chain quality.
\end{theorem}
\begin{proof}
  \qed
\end{proof}
