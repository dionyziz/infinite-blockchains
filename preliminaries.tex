\section{Preliminaries}

\noindent
\textbf{The backbone model.}
We adapt the Bitcoin Backbone model~\cite{EC:GarKiaLeo15} to infinihe executions. Here we give a brief
overview of the elements of the Bitcoin Backbone model we will make use of.
We consider a setting in which $n$ parties (numbered $1$ through $n$) collaborate to establish consensus on
an ever-growing ledger. Among them, $t$ are controlled by an adversary. We model
the parties as Interactive Turing Machines which are controlled by an
environment $\mathcal{E}$ (akin to the role of the environment in the Universal
Compasibility~\cite{EPRINT:Canetti07} model). Our treatment is synchronous: time
% TODO: Check UC reference
is split into discrete \emph{rounds}. A round constitutes sufficient time for a
network message to reach the whole network. At the beginning of each round, the
environment activates every party in order. At the end of the round, the party
can broadcast a message. The message broadcast by each parties during the
previous round is received by all parties at the beginning of the next round.
The adversary has a chance to see the messages broadcast during a round before
she can execute her attack, and is therefore a \emph{rushing adversary}. The
adversary can modify the order of message delivery and inject arbitrary
messages, as long as no messages are lost. Hence, we assume that the network is
reliable but anonymous (i.e., Sybil attackable~\cite{DBLP:conf/iptps/Douceur02}).
The adversary is allowed to perform different reorderings and message injections
for each receipient in an effort to make them disagree with one another.
The parties share a reference string, the genesis block $\mathcal{G}$, prior to
which no mining has occurred.
Each honest party maintains a local chain $\chain$ which begins with the Genesis
block. The party attempts to \emph{mine} a new block on top of the currently
adopted chain. Each party has $q \in \mathbb{N}$ attempts to mine a block during
a round, each attempt of which has a constant probability $p$ of succeeding
(the constant probability assumption corresponds to constant \emph{difficulty}
in the polynomial setting, but the notion of a mining target and difficulty are
ill-defined in infinite executions). The adversary, collectively through its $t$
parties, can make $tq$ mining attempts per round.
The newly mined block contains transactions that the miner introduces into the
\emph{data} portion of the block. If an honest party successfully mines a block,
she extends her local chain with $\chain$ the new block $B$ to obtain $\chain B$
and subsequently \emph{broadcasts} the extended chain. While honest parties
immediately broadcast blocks, the adversary can choose
to withhold blocks and broadcast them at a later time, a strategy which is
useful for various attacks including the infamous
\emph{selfish mining attack}~\cite{FC:EyaSir14}. When an honest party receives a
new candidate chain from the network, she compares its \emph{length} against the
local chain and \emph{adopts} the chain from the network in case it is longer.

\noindent
\textbf{Chain Traversal.}
Chains are sequences of blocks. We will adopt the following notation for
referring to blocks within a chain. Let $\chain[i]$ indicate the $i^\text{th}$
block in the chain, where $\chain[0]$ is the first block of the chain. Every
valid chain must begin with the genesis block, therefore it will have $\chain[0]
= \mathcal{G}$. Let $\chain[-i]$ indicate the $i^\text{th}$ block from the end.
We will call the most recent block of the chain, i.e., $\chain[-1]$, the
\emph{tip}. Let the range notation $\chain[i{:}j]$ indicate the subsequence of
consecutive blocks in $\chain$ starting from the $i^\text{th}$ block (inclusive)
and ending in the $j^\text{th}$ block (exclusive). Range indices can also be
negative. Omitting the start index of the range is taken to mean that the
subsequence is to begin at the beginning of the chain, while omitting the end
index of the range is taken to mean that the subsequence is to end at the end of
the chain. Therefore, $\chain[:k]$ indicates the first $k$ blocks of the chain,
while $\chain[-k:]$ indicates the last $k$ blocks of the chain.

The indexing notation should not be confused with $[n]$ which indicates the set
of all integers from $1$ (inclusive) to $n$ (inclusive). We will treat
mathematical functions as sets of tuples from the input to the output. Hence, if
$f = \{(1, 2), (3, 4)\}$, then $f(1) = 2$ and $f(3) = 4$. We will use $|\chain|$
to indicate the length of a sequence and $\epsilon$ to indicate the empty
sequence. We denote concatenation of a sequence $\chain$ with an element $b$
using $\chain b$. We use $\aleph_0$ for the cardinality of the natural numbers.
