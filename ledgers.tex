\section{Desirable Properties of Ledgers}

We will now study executions which are \emph{unbounded}, i.e., they never halt
and that \emph{make progress}, i.e., advance rounds in perpetuity. As such, in
these executions, every round number $r \in \mathbb{N}$ will eventually appear.

The protocol which we will build allows the environment to ask an honest party
to inject a transaction into the next block that they are mining. The honest
party protocol then provides a method to \emph{read} the ledger of the party,
which returns a sequence of transactions. Let $\ledger_\party^r$ denote the
ledger reported by party $\party$ at round $r$. While at every round every
party's reported ledger is finite, we can take the ledgers to the limit and
define the \emph{eventually adopted ledger} of a party.

\begin{definition}[Eventually adopted ledger]
  Let $\party$ be a party in some execution. We define the \emph{eventually
  adopted ledger} $\ledger_\party^\infty$ of the party to be
  a (potentially countably infinite) sequence of transactions as follows. The
  $i^\text{th}$ element of $\ledger_\party^\infty$ is defined to be $\tx$
  \emph{iff} there exists a round $r \in \mathbb{N}$ such that for every round
  $r' > r$, it holds that $\ledger_\party^r[i] = \tx$.
\end{definition}

The above definition includes a transaction in the eventually adopted ledger if
its position stabilizes for ever after a certain round. Note that ill-behaving
blockchain protocols can give rise to eventually adopted ledgers which are
finite; or which are countably infinite but partially defined; or which are
different for every party. It is desired that the eventually adopted ledgers of
all parties are infinite, in agreement, and total. Let us now provide
definitional extensions of the known properties that a distributed ledger must
have.

\begin{definition}[Persistence]
  A ledger protocol has \emph{persistence} if for every honest parties
  $\party_1, \party_2$ it holds that
  $\ledger_{\party_1}^\infty = \ledger_{\party_2}^\infty$. We use
  $\ledger^\infty$ to refer to that common ledger. Furthermore, we require that
  $\ledger^\infty$ is total.
\end{definition}

\begin{definition}[Liveness]
  A ledger protocol has \emph{liveness} if, whenever all honest parties attempt
  to inject a transaction $\tx$ for every round after a given round
  $r \in \mathbb{N}$, the transaction appears in $\ledger_\party^\infty$ for all
  $\party$.
\end{definition}

It follows that, if an infinite distributed ledger protocol has persistence and
liveness and is used in an execution where fresh transactions always appear
anew in the environment in perpetuity, then $\ledger^\infty$ will be
everywhere well-defined and $|\ledger^\infty| = \aleph_0$.
