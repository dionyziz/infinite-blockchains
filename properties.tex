\section{Desirable Properties of Chains}

Let $\chain^r_\party$ denote the chain adopted by the honest party $\party$ at
the end of round $r$. We are now ready to define eventually adopted chains.

\begin{definition}[Eventually adopted chains]
  Let $\party$ be an honest party. The \emph{eventually adopted chain}
  $\chain^\infty_\party$ is defined as the sequence of blocks such that the
  $i^\text{th}$ block of $\chain^\infty_\party$ is $B$ \emph{iff} there exists
  some round $r \in \mathbb{N}$ such that for every round $r' > r$ it holds that
  $\chain^r_\party[i] = B$.
\end{definition}

We can think of the above chain as the chain adopted at the end of (infinitely
running) time. Blocks that belong to this chain are deemed \emph{stable}. While
each party never arrives at a definite conclusion that a block is stable
(which means that the protocol remains only of theoretical interest),
nevertheless we can argue about stable blocks mathematically.

We now state three definitional extensions for the properties of chains.

\begin{definition}[Chain growth]
  A blockchain protocol satisfies \emph{chain growth} if for every honest
  party $\party$ and for every round $r$, there exists some round $r' > r$ such
  that $|\chain^{r'}_\party| > |\chain^r_\party|$.
\end{definition}

\begin{definition}[Common prefix]
  A blockchain protocol satisfies \emph{common prefix} if for every honest
  parties $\party_1, \party_2$ it holds that
  $\chain^\infty_{\party_1} = \chain^\infty_{\party_2}$. We will refer to this
  common chain as $\chain^\infty$.
\end{definition}

If our protocol satisfies both the Chain growth and Common prefix properties,
then it follows that $\chain^\infty$ is well-defined and
$|\chain^\infty| = \aleph_0$.

\begin{definition}[Chain quality]
  A blockchain protocol satisfies \emph{chain quality} if for every honest party
  $\party$ and for every $k \in \mathbb{N}$ it holds that at least one block in
  $\chain^\infty_\party[k:]$ has been honestly generated.
\end{definition}

From these definitions, it follows that, if our blockchain protocol has Common
prefix, then the distributed ledger protocol arising from it has Persistence.
Likewise, if our blockchain protocol has Chain growth and Chain quality, then
the distributed ledger protocol arising from it has Liveness.
